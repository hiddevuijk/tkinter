%%%%%%%%%%%%%%%%%%%%%%%%%%%%%%%%%%%%%%%%%
% Short Sectioned Assignment
% LaTeX Template
% Version 1.0 (5/5/12)
%
% This template has been downloaded from:
% http://www.LaTeXTemplates.com
%
% Original author:
% Frits Wenneker (http://www.howtotex.com)
%
% License:
% CC BY-NC-SA 3.0 (http://creativecommons.org/licenses/by-nc-sa/3.0/)
%
%%%%%%%%%%%%%%%%%%%%%%%%%%%%%%%%%%%%%%%%%

%----------------------------------------------------------------------------------------
%	PACKAGES AND OTHER DOCUMENT CONFIGURATIONS
%----------------------------------------------------------------------------------------

\documentclass[paper=a4, fontsize=11pt]{scrartcl} % A4 paper and 11pt font size

\usepackage[T1]{fontenc} % Use 8-bit encoding that has 256 glyphs
\usepackage{fourier} % Use the Adobe Utopia font for the document - comment this line to return to the LaTeX default
\usepackage[english]{babel} % English language/hyphenation
\usepackage{amsmath,amsfonts,amsthm} % Math packages

\usepackage{lipsum} % Used for inserting dummy 'Lorem ipsum' text into the template

\usepackage{sectsty} % Allows customizing section commands
\allsectionsfont{\centering \normalfont\scshape} % Make all sections centered, the default font and small caps

\usepackage{fancyhdr} % Custom headers and footers
\pagestyle{fancyplain} % Makes all pages in the document conform to the custom headers and footers
\fancyhead{} % No page header - if you want one, create it in the same way as the footers below
\fancyfoot[L]{} % Empty left footer
\fancyfoot[C]{} % Empty center footer
\fancyfoot[R]{\thepage} % Page numbering for right footer
\renewcommand{\headrulewidth}{0pt} % Remove header underlines
\renewcommand{\footrulewidth}{0pt} % Remove footer underlines
\setlength{\headheight}{13.6pt} % Customize the height of the header

\numberwithin{equation}{section} % Number equations within sections (i.e. 1.1, 1.2, 2.1, 2.2 instead of 1, 2, 3, 4)
\numberwithin{figure}{section} % Number figures within sections (i.e. 1.1, 1.2, 2.1, 2.2 instead of 1, 2, 3, 4)
\numberwithin{table}{section} % Number tables within sections (i.e. 1.1, 1.2, 2.1, 2.2 instead of 1, 2, 3, 4)

\setlength\parindent{0pt} % Removes all indentation from paragraphs - comment this line for an assignment with lots of text

%----------------------------------------------------------------------------------------
%	TITLE SECTION
%----------------------------------------------------------------------------------------

\newcommand{\horrule}[1]{\rule{\linewidth}{#1}} % Create horizontal rule command with 1 argument of height

\title{\vspace{-3cm}
\huge Assignment 3 \\ % The assignment title
\horrule{2pt} \\ % Thick bottom horizontal rule
\vspace{-2cm}
}

\author{} % Your name

\date{} % Today's date or a custom date

\begin{document}

\maketitle % Print the title

%----------------------------------------------------------------------------------------
%	PROBLEM 1
%----------------------------------------------------------------------------------------
\section*{Problem 1: Number Fluctuations and Isothermal compressibility}

For the derivation of the compressibility equation we needed the relation between the fluctuations in the number of
particles and the isothermal compressibility:
\begin{align*}
 \frac{\overline{N^2} - \overline{N}^2}{\overline{N}} = \rho k T \kappa
 \ \ \ \text{with} \ \ \ 
 \kappa \equiv - \frac{1}{V} \left(\frac{\partial V}{\partial P}\right)_{T,N},
\end{align*}
where $\kappa$ is called the isothermal compressibility.



A) \hspace{.5cm} The first step is to express the number fluctuations as some partial derivative. Later you can 
relate that derivative to $\left( \frac{\partial V}{\partial P} \right)_{T,N}$ by using thermodynamic relations.
The average of a function $A(N,j)$ in a grand canonical ensemble is
\begin{align*}
 \overline{A} = \sum_{N,j} A(N,j) P_{N,j},
\end{align*}
where $j$ labels the energy state, and 
\begin{align*}
 P_{N,j} = \frac{e^{-\beta E_{N,j}+\beta \mu N}}{\Xi(V,\beta,\mu)}
\ \ \ \ \text{and} \ \ \ \ \
\Xi(V,\beta,\mu)  = \sum_{N,j} e^{-\beta E_{N,j}+\beta \mu N}.
\end{align*}

Show that
\begin{align*}
 \overline{N^2}  = kT \left( \frac{\partial \overline{N}}{\partial \mu}\right)_{T,V} + 
 \overline{N}^2.
\end{align*}
So that in the thermodynamic limit
\begin{align*}
  \overline{N^2}- \overline{N}^2  = kT \left( \frac{\partial N}{\partial \mu}\right)_{T,V}.
\end{align*}



B) \hspace{.5cm}  The number fluctuations are now expressed using a derivative with respect to the chemical potential, 
but we want a derivative with volume and pressure with constant T and N. So we need to change the derivative with
respect to the chemical potential to a derivative with respect to the pressure, and then we can use the cyclic rule 
for partial
derivatives to relate $\left( \frac{\partial N}{\partial P}\right)_{T,V}$ to the desired derivative.

The thermodynamic potential corresponding to the grand canonical ensemble is the grand potential:
\begin{align*}
 \Omega = -PV \equiv A - \mu N,
\end{align*}
where $A$ is the Helmholtz free energy.
Use this to show that
\begin{align*}
 \left( \frac{\partial N}{\partial \mu} \right)_{T,V} = \frac{N}{V} \left( \frac{\partial N}{\partial P} \right)_{T,V}.
\end{align*}

C) \hspace{.5cm} Show that
\begin{align*}
 \left( \frac{\partial N}{\partial \mu} \right)_{V,T} = - \rho^2 \left( \frac{\partial V}{\partial P} \right)_{N,T},
\end{align*}
and use it to show that
\begin{align*}
 \frac{\overline{N^2} - \overline{N}^2}{\overline{N}} =
 - \frac{kT\rho}{V} \left(\frac{\partial V}{\partial P}\right)_{T,N}.
\end{align*}
You can use the cyclic property of partial derivatives,
\begin{align*}
 \left(\frac{N}{P} \right)_{V,T}  \left(\frac{V}{N} \right)_{P,T}  \left(\frac{P}{V} \right)_{N,T} = -1,
\end{align*}
and $V=v N_a N$, where $v$ is the molar volume (the volume of one mole of gas at a given T and P)
and $N_a$ is Avogadro's number.



%----------------------------------------------------------------------------------------
%	PROBLEM 2
%----------------------------------------------------------------------------------------

\section*{Problem 2: Fourier transformation of the Ornstein-Zernike Equation}

Fourier transform the Ornstein-Zernike equation,
\begin{align*}
 h(|\mathbf{r}_1 - \mathbf{r}_2|)=c(|\mathbf{r}_1-\mathbf{r}_2|) +
 \rho \int d\mathbf{r}_3 c(|\mathbf{r}_1-\mathbf{r}_3|)  h(|\mathbf{r}_2 - \mathbf{r}_3|),
\end{align*}
and show that
\begin{align*}
 \hat{C}(\mathbf{k}) = \frac{\hat{H}(\mathbf{k})}{1+\rho \hat{H}(\mathbf{k})}
\ \ \ \ \text{and} \ \ \ 
 \int d\mathbf{r} h(|\mathbf{r}|) = \hat{H}(0).
\end{align*}
Use the following definition of the Fourier transformation of a function $f(\mathbf{r})$ with $r\in \mathbb{R}^3$:
\begin{align*}
 \hat{F}(\mathbf{k}) = \int d^3 r \ \  e^{i \mathbf{k} \cdot \mathbf{r}} f(\mathbf{r}) \ \ \ \ \ \ \
  f(\mathbf{r}) = \int \frac{d^3k}{(2 \pi)^3}\ \  e^{-i \mathbf{k} \cdot \mathbf{r}} \hat{F}(\mathbf{k}) .
\end{align*}




%----------------------------------------------------------------------------------------
%	PROBLEM 3
%----------------------------------------------------------------------------------------

\section*{Problem 3: The equation of state of hard spheres}
Use the pressure equation,
\begin{align*}
 \frac{P}{k T} = \rho - \frac{\rho^2}{6kT} \int_0^\infty dr 4 \pi r^3 u'(r)g(r),
\end{align*}
to derive the equation of state for hard spheres:
\begin{align*}
 \frac{P}{k T} = \rho - \frac{2 \pi \rho^2 \sigma^3}{3} g(\sigma_+),
\end{align*}
where $\sigma_+=\lim_{\delta\to 0} \sigma + \delta$. 
The hard sphere potential is
\begin{align*}
u_{hs} = 
 \begin{cases}
  \infty,& \ \ \ \text{if}\ \  r<\sigma\\
   0\ \ ,& \ \ \ \text{if}\ \  r>\sigma.
 \end{cases}
\end{align*}
Note that neither of the cases contains $r=\sigma$ because only the limiting cases are well defined.
So, for example, $\lim_{r\to\sigma_+}$ belongs to the $r>\sigma$ case.

The problem with the integral is the derivative of the potential because the potential has an infinite discontinuity
at $\sigma$. It is possible to write the integrant as some function times a derivative of an other 
function which has a finite discontinuity at $\sigma$.
To do this you need to define two new functions:
\begin{align*}
 y(r) = & e^{\beta u(r)}g(r), \\
 e(r) = & e^{-\beta u(r)}.
\end{align*}



%----------------------------------------------------------------------------------------

\end{document}