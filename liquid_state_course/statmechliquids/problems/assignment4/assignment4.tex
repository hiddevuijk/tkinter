%%%%%%%%%%%%%%%%%%%%%%%%%%%%%%%%%%%%%%%%%
% Short Sectioned Assignment
% LaTeX Template
% Version 1.0 (5/5/12)
%
% This template has been downloaded from:
% http://www.LaTeXTemplates.com
%
% Original author:
% Frits Wenneker (http://www.howtotex.com)
%
% License:
% CC BY-NC-SA 3.0 (http://creativecommons.org/licenses/by-nc-sa/3.0/)
%
%%%%%%%%%%%%%%%%%%%%%%%%%%%%%%%%%%%%%%%%%

%----------------------------------------------------------------------------------------
%	PACKAGES AND OTHER DOCUMENT CONFIGURATIONS
%----------------------------------------------------------------------------------------

\documentclass[paper=a4, fontsize=11pt]{scrartcl} % A4 paper and 11pt font size

\usepackage[T1]{fontenc} % Use 8-bit encoding that has 256 glyphs
\usepackage{fourier} % Use the Adobe Utopia font for the document - comment this line to return to the LaTeX default
\usepackage[english]{babel} % English language/hyphenation
\usepackage{amsmath,amsfonts,amsthm} % Math packages

\usepackage{lipsum} % Used for inserting dummy 'Lorem ipsum' text into the template

\usepackage{sectsty} % Allows customizing section commands
\allsectionsfont{\centering \normalfont\scshape} % Make all sections centered, the default font and small caps

\usepackage{fancyhdr} % Custom headers and footers
\pagestyle{fancyplain} % Makes all pages in the document conform to the custom headers and footers
\fancyhead{} % No page header - if you want one, create it in the same way as the footers below
\fancyfoot[L]{} % Empty left footer
\fancyfoot[C]{} % Empty center footer
\fancyfoot[R]{\thepage} % Page numbering for right footer
\renewcommand{\headrulewidth}{0pt} % Remove header underlines
\renewcommand{\footrulewidth}{0pt} % Remove footer underlines
\setlength{\headheight}{13.6pt} % Customize the height of the header

\numberwithin{equation}{section} % Number equations within sections (i.e. 1.1, 1.2, 2.1, 2.2 instead of 1, 2, 3, 4)
\numberwithin{figure}{section} % Number figures within sections (i.e. 1.1, 1.2, 2.1, 2.2 instead of 1, 2, 3, 4)
\numberwithin{table}{section} % Number tables within sections (i.e. 1.1, 1.2, 2.1, 2.2 instead of 1, 2, 3, 4)

\setlength\parindent{0pt} % Removes all indentation from paragraphs - comment this line for an assignment with lots of text

%----------------------------------------------------------------------------------------
%	TITLE SECTION
%----------------------------------------------------------------------------------------

\newcommand{\horrule}[1]{\rule{\linewidth}{#1}} % Create horizontal rule command with 1 argument of height

\title{\vspace{-3cm}
\huge Assignment 4 \\ % The assignment title
\horrule{2pt} \\ % Thick bottom horizontal rule
\vspace{-2cm}
}

\author{} % Your name

\date{} % Today's date or a custom date

\begin{document}

\maketitle % Print the title



%----------------------------------------------------------------------------------------
%	PROBLEM 1
%----------------------------------------------------------------------------------------

\section*{Problem 1: The density expansion of the PY equation}

The Percus-Yevick equation is
\begin{align*}
 y_{12} = 1+\rho \int dr_3 f_{13}y_{13}h_{23},
\end{align*}
where $y_{12} \equiv y(|\mathbf{r}_1-\mathbf{r}_2|)$ and the same for the other functions. The function $h$ is related 
to $y$ by $y(r)=e^{-\beta u(r)} \left[ h(r) +1 \right]$. You can use the expression for $y_{12}$ to replace the 
functions $y$ on the RHS of the equation. Repeating this procedure generates a serie expansion of $y_{12}$ in powers of
$\rho$ (this is called a Neumann series). Only the last term in this series depends on $y$ and for a small density you 
can truncate the series somewhere, which results in a closed approximate equation for $y$.
\\

A)\hspace{.5cm}
Write down the series solution for $y_{12}$ of the Percus-Yevick equation including the $\rho^2$ term and neglecting the
the higher order term.
Your result should be writen as integrals over $f$ functions.
\\
B)\hspace{.5cm}
Replace the $f$ functions by the corresponding diagrams.
\\

Because the result uses the PY-approximation you will not get all the diagrams for the order $\rho^2$ term that was
shown in the lectures.
%----------------------------------------------------------------------------------------

\end{document}